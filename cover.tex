% !Mode:: "TeX:UTF-8"

\newcommand{\chinesethesistitle}{密集热点区域无线网络性能分析与优化} %授权书用,无需断行
\newcommand{\englishthesistitle}{\uppercase{performance analysis and optimization in dense hot point wireless networks}} %\uppercase作用:将英文标题字母全部大写;
\newcommand{\chinesethesistime}{2018~年~6~月}  %封面底部的日期中文形式
\newcommand{\englishthesistime}{June, 2018}    %封面底部的日期英文形式

\ctitle{密集热点区域无线网络性能分析与优化}  %封面用论文标题,自己可手动断行
\cdegree{\cxueke\cxuewei}
\csubject{电子与通信工程}                 %(~按二级学科填写~)
\caffil{电子与信息工程学院} %(在校生填所在系名称,同等学力人员填工作单位)
\cauthor{麻津铭}
\csupervisor{陈晓华教授} %导师名字
%\cassosupervisor{副导名}%若没有,请屏蔽掉此句。
%\ccosupervisor{联导名}%若没有,请屏蔽掉此句。


\cdate{\chinesethesistime}

\etitle{\englishthesistitle}
\edegree{\exuewei \ of \exueke}
\esubject{Electronics and Communication Engineering}  %英文二级学科名
\eaffil{School of Electronics and \qquad \quad Information Engineering}
\eauthor{Ma Jin-Ming}                   %作者姓名 (英文)
\esupervisor{Prof. Hsiao-Hwa Chen}       % 导师姓名 (英文)
%\eassosupervisor{Prof. Assosuper}%若没有,请屏蔽掉此句。
%\ecosupervisor{Prof. Cosuper}%若没有,请屏蔽掉此句。
\edate{\englishthesistime}

\natclassifiedindex{TN929.53}  %国内图书分类号
\internatclassifiedindex{621.3}  %国际图书分类号
\statesecrets{公开} %秘密

\iffalse
\BiAppendixChapter{摘~~~~要}{}  %使用winedt编辑时文档结构图(toc)中为了显示摘要,故增加此句;
\fi
\cabstract{
第五代移动通信技术(~the 5~th Generation Communication Technology,5G~)将于2020年前后开始商用,
相比于第四代移动通信技术(~the 4th Generation Communication Technology,4G~),
5G~网络的容量需求将有~1000~倍的增长。
超密集组网(~Ultra Dense Networks, ~UDNs~)通过在区域内放置更多的微基站(~Small Base Station,SBS~)换取网络性能的提升,
是一种能够提供巨大容量增益的十分可靠的技术。
但是,超密集组网使~SBS~之间的距离更近,随之带来的~SBS~之间干扰问题越发明显。
在超密集组网场景中,由于~SBS~之间的距离较近,因此~SBS~和用户构成的通信链路的信道状态信息(~Channel State Information,CSI~)
易于汇总到控制端,再由控制端控制~SBS~进行数据的发送,达到~SBS~之间协作的目的。
由于协作的~SBS~的天线之间的距离远远大于天线的相关距离(~Coherence Distance~),
因此协作的~SBS~可以构成一个分布式的多输入多输出(~Multiple-Input Multiple-Output,MIMO~)系统,
协作的~SBS~所服务的用户可以通过波束成型(Beamforming,BF)的方法将用户的接收信号向量在空域上相互正交,
达到干扰消除的目的。

首先,本文对基于格的基站部署方法进行了研究,包括环形的基站部署方法和方格点的基站部署方法。
对上述两种基站部署方法的拓扑结构进行了介绍,并对性能进行了分析。
对于环形的基站部署方法,讨论了环的半径~$R$~和基站的个数~$N$~对整个网络的性能的影响。
对于方格点的基站部署方法,讨论了~SBS~之间的横向间距~$l_1$~和纵向间距~$l_2$~对网络的性能的影响。

接着,本文对基于泊松点过程(~Poisson Point Process,PPP~)的基站部署的性能进行了研究,
并假设用户的分布为混合二维高斯分布以反映网络中的用户的不均匀性和聚集性。
提出了一种简单可行的办法计算该场景下的覆盖率和单位面积频谱效率等性能指标,
并讨论了影响网络性能的相关参量。

本文最后对网络中用户的接收信干比(~Signal Inference Ratio,SIR~)性能做进一步的优化。
优化的过程分为两步,
第一步出于对网络的同步性和复杂度要求的考虑本文提出了基于深度优先搜索和以用户为中心的基站分簇算法,
第二步对簇内采用基于~SBS~间协作的分布式迫零预编码(~Zero-Forcing Beamforming,ZFBF~)技术优化网络中用户的接收~SIR,
应用仿真分析的方法讨论了优化后的网络的性能。
}
\ckeywords{超密集组网;随机几何;微基站分簇;联合传输;波束形成}

\eabstract{

The 5th-generation mobile communication technology(5G) will begin commercial application around 2020.
Compared to the 4th-generation mobile communication technology(4G),
the capacity requirement of the 5G network will increase by 1,000 times.
The ultra-dense networks(UDNS) improves the network performance by placing more base stations in the area.
It is a very reliable technology that can provide huge capacity gains.
However, the UDNs makes the distance between SBSs closer,
and the interference between ~SBSs~ gets more obvious.
In UDNs, since the distance between SBSs is relatively close,
 it is easier to get the channel state information (CSI) of the communication link formed by SBS
 and the user and submit to the controler.
The controler instructs SBSs to send data, achieving the cooperation between SBSs.
Since the distance between the antennas of the cooperative SBSs is much larger than the coherence distance of the antennas,
the cooperative SBSs can form a distributed multiple-input multiple-output (MIMO) system.
The users of the cooperative SBS can use the beamforming(BF) method to orthogonalize the user's received signal vectors in the space domain,
achieving the purpose of interference cancellation.

Firstly,the grid-based base station deployment methods are introduced, including the ring-shaped base station deployment method and the grid point base station deployment method.
The topology structure of the above two base station deployment methods is introduced and the performance is analyzed.
For the ring-shaped base station deployment method, the effect of the ring radius $R$ and the number of base stations $N$ on the performance of the entire network is discussed.
For the grid points base station deployment method,
the influence of the horizontal spacing between SBSs $l_1$ and the vertical spacing $l_2$ on the performance of the network is discussed.

Next, the performance of base station deployments based on the Poisson Point Process(PPP) is discussed.
It is assumed that the user's distribution is a mixed two-dimensional Gaussian distribution to reflect the user's heterogeneity and aggregation in the network.
A tractable method is proposed to calculate the coverage area rate and spectral efficiency  under this scenario.
And related parameters that affect network performance are discussed.

Finally, the user's receive signal-to-interference ratio (SIR) performance in the network is optimizated.
The optimization process is divided into two steps.
At the first step, considering the requirements of synchronization and complexity of the network,
clustering based on th depth-first search and user-centric clustering algorithm are proposed.
At the second step is  the Zero-Forcing Beamforming (ZFBF) technique based on SBS cooperation are used to optimize the user's receive~SIR in the network,
besides, simulation analysis is used to discuss the performance of the optimized network.
}

\ekeywords{UDNs, Stochastic Geometry, Clustering, CoMP, Beamforming}


\makecover
\clearpage
\begin{table}[htbp]
\label{cluster_zfbf_sim_para}
\vspace{0.5em}\centering\wuhao
\begin{tabular}{cccc}
\toprule[1.5pt]
参量 & & & 设置 \\
\midrule[0.5pt]
基站的分布~$\Phi$~ & & & 泊松点过程 \\
用户的分布~$\Psi$~  & & & 混合二维高斯分布\\
用户的分散程度~$\sigma$~ & & &  ~$5\mathrm{m}$~ \\
区域~$\mathcal{D}$~的大小  & & & ~$100\mathrm{m} \times 100 \mathrm{m}$~ \\
微基站的天线数 & & & 1 \\
基站的最大发射功率~$P$~ & & & 1 \\
用户的天线数 & & & 1 \\
服务基站的选择方式 & & & 簇内多用户联合传输 \\
\bottomrule[1.5pt]
\end{tabular}
\end{table}

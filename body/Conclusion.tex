% !Mode:: "TeX:UTF-8"

\BiAppendixChapter{结\quad 论}{Conclusions}
本文首先给出了不同信道环境不同调制方式下的~Shannon~限,从理论上说明了编码在不同信道下误码率性能有差异的原因;其次,将~LDPC~码与~MIMO~ 系统相结合,提出一种基于~LDPC~码的~MIMO~分集系统,并且利用~LDPC~码信息位与校验位的对应关系,提出一种基于信号处理和信息传递的软译码算法;最后分别对时间选择性衰落信道、频率选择性衰落信道和时频双选信道下的系统性能进行了仿真分析。具体分为如下几个方面:

(1)~~给出了高斯信道、非频选快衰落信道和非频选慢衰落信道下,不同调制方式的信道容量和~Shannon~限。对于高斯信道,本文给出了适用于不同调制方式的信道容量通用表达式,并且给出了常见的几种调制方式(PAM、PSK、QAM~和~正交信号调制)的信道容量和~Shannon~限。在非频选快衰落信道中,本文假设接收机完全已知信道状态信息,给出了不同调制方式的信道容量和~Shannon~限。针对非频选慢衰落信道,借鉴中断概率的概念和计算方法,对非频选慢衰落信道建立了数学模型,将信道容量分为连续和离散两部分,并对其容量和~Shannon~限进行了推导和计算。结果表明,在调制方式和信噪比相同的前提下,高斯信道的容量最大,快衰落信道次之,慢衰落信道最小;在调制方式和码速率相同的前提下,高斯信道~Shannon~限最小,快衰落信道~Shannon~限次之,慢衰落信道~Shannon~限最大。也就是说要想达到无误传输,高斯信道所需要的信噪比最小,而慢衰落信道需要的最大,这也从理论上说明了信道编码在慢衰落信道中误码率性能差的原因;在时间选择性衰落信道中,衰落系数变化越快,信道容量越大,Shannon~限越小,从侧面证明了交织在信道编码系统中是非常有效的提高可靠性的手段。

(2)~~提出一种基于~LDPC~码的~MIMO~分集系统。分别考虑时间选择性衰落信道和频率选择性衰落信道,以时间分集~$L = 2$、$L = 3$~和空间分集~$2 \times 2$~MIMO~为例,对基于~LDPC~码的~MIMO~分集系统的原理和实现方案进行了详细介绍,由于本文使用的是码速率为~$1/2$~的~LDPC~码,因此,与传统分集系统相比,传输效率完全一样,但是误码率性能有很大的提升;此外,给出了不同信道下,不同分集系统中~LDPC~码的~LLR~计算方法。仿真结果表明,在时间选择性衰落信道下,随着信道衰落速度的增加,本文提出的分集系统误码率性能越来越好,且性能优于传统分集系统 ,同时随着信噪比的增加,误码率下降速度加快;在频率选择性衰落信道下,当信噪比较大时,本文提出的分集系统误码率性能明显好于传统分集系统,且随着信噪比的增加性能增益越来越大;在时频双选信道下,本文提出的分集系统具有更优异的误码率性能。

(3)~~提出一种基于信号处理和信息传递的软译码算法。由于本文使用的是码速率为~$1/2$~的~LDPC~码,其信息位和校验位的长度相同,存在唯一映射关系,考虑这一对应关系和不同信道之间相互独立的特性,提出一种基于信号处理和信息传递的软译码算法,这一软译码算法将信息位和校验位之间的关系看作是~SPC~ 码,仿照~SPC~码之间的信息传递过程,对~LDPC~码信息位和校验位的~LLR~进行了合并和修正。仿真结果表明,在时间选择性衰落信道下、频率选择性衰落信道和时频双选衰落信道下,与直接计算~LLR~译码相比,这一软译码算法都可以有效降低系统误码率。

本文还存在一些问题尚未解决。比如在时间分集~$L > 2$~ 时,实际上可以根据分集增益的大小,来设计码速率为~$1/N$~的~LDPC~ 码,还可以采用与~Turbo~码级联的方式。这些问题可以作为后续研究的方向。






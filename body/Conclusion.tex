% !Mode:: "TeX:UTF-8"

\BiAppendixChapter{结\quad 论}{Conclusions}

本文首先根据提出的密集热点区域无线网络的网络定义,
对密集热点区域无线网络进行建模,
并对该场景下的无线网络进行了性能分析,
性能指标主要为网络的遍历容量,
网络的覆盖率和网络的单位面积频谱效率。
然后根据网络的特性,提出了一种干扰管理的网络架构对网络的性能进行了优化。
网络的架构主要分为两个子部分,即首先对网络中的微基站分簇,
采用基于~ZFBF~的多用户联合传输技术实现对簇内的干扰进行消除。具体分为如下两个方面:

(1)~~根据密集热点区域无线网络的基站的拓扑结构,用户的统计特性,信道特性和网络架构对网络进行建模,
得到网络的信干噪比,遍历容量的表达式。
对密集热点区域无线网络的性能进行分析,性能的参数指标主要为网络的遍历容量,
网络的覆盖率和网络的单位面积频谱效率,得到了易于计算的网络覆盖率和网络的单位面积频谱效率的表达式,
对网络中各个位置的遍历容量,网络的覆盖率和网络的单位面积频谱效率进行了仿真和数值分析,
验证了理论分析的正确性。

(2)~~对密集热点区域无线网络进行优化,根据对密集热点区域无线网络的性能分析可知,
网络中的边缘用户受到微基站的强烈干扰,因此边缘用户的性能较差。
为了减少边缘用户,优化网络的性能,提出了网络干扰管理架构,该干扰管理架构分为两个部分,
第一个部分对基站进行分簇,提出两种微基站的分簇算法,
分别为基于深度优先搜索的微基站分簇算法和基于~k~-~均值的微基站分簇算法。
第二个部分对簇内进行干扰消除,提出采用基于~ZFBF~的簇内干扰消除算法,
该算法通过将待发送的信号在空域中对齐的方法对簇内的用户实现干扰消除。
对提出的干扰管理优化算法的性能进行仿真分析,得到的结果表明,应用了干扰管理算法的网络覆盖率性能
有明显的提升,由于单位面积频谱效率是覆盖率的积分形式,该结果也表明应用干扰管理算法对网络的性能进行优化,
也大大提升了网络的单位面积频谱效率。

本文还存在一些问题尚未解决。可以根据用户的位置和信道状态信息对簇内用户进行用户选择,进一步提升网络的性能。
可以增加网络的基站睡眠与唤醒机制,降低用户的干扰和网络的总的能耗。这些问题可以作为后续的研究方向。

% !Mode:: "TeX:UTF-8"

\BiAppendixChapter{结\quad 论}{Conclusions}

本文针对~5G~中提出的的超密集组网,选择超密集组网环境下的一种特定的网络场景。
定义密集热点区域无线网络场景,
该场景中假设基站的部署服从泊松点过程,
该假设符合当前蜂窝网络场景的实情,
因为当前移动通信网络中,
网络中存在很多微基站和家庭基站,
基站的拓扑结构呈现出随机性。
考虑到用户在区域内分布的不均匀性,
因为微基站服务的区域多为用户需求量大,容量需求高的地带,
而远离微基站的位置用户的需求量较小,
针对该特性,假设用户服从以基站位置为均值的二维高斯分布,
其中方差参数~$\sigma$~用来表示用户的分散程度,
该假设很好的反应了微基站所服务的用户的统计特性。

本文首先根据提出的密集热点区域无线网络的网络定义,
对密集热点区域无线网络进行建模,
并对该场景下的无线网络进行了性能分析,
性能指标主要为网络的遍历容量,
网络的覆盖率和网络的单位面积频谱效率。
然后根据网络的特性,提出了一种干扰管理的网络架构对网络的性能进行了优化。
网络的架构主要分为两个子部分,即首先对网络中的微基站分簇,
在~CRAN~架构下采用基于~ZFBF~的多用户联合传输技术实现对簇内的干扰进行消除。具体分为如下两个方面:

(1)~~根据密集热点区域无线网络的基站的拓扑结构,用户的统计特性,信道特性和网络架构对网络进行建模,
得到网络的信干噪比,遍历容量的表达式。
对密集热点区域无线网络的性能进行分析,性能的参数指标主要为网络的遍历容量,
网络的覆盖率和网络的单位面积频谱效率,得到了易于计算的网络覆盖率和网络的单位面积频谱效率的表达式,
对网络中各个位置的遍历容量,网络的覆盖率和网络的单位面积频谱效率进行了仿真和数值分析,
验证了理论分析的正确性。
网络对不同位置的遍历容量进行仿真,得到了区域内遍历容量的热力分布图,
热力分布图表明处于微基站边缘的用户由于受到了其他微基站的强烈干扰,遍历容量较处于微基站中心区域的用户相差较大,
同时信道的大尺度衰落系数为~4~的结果远远好于大尺度衰落系数为~2~时的结果,
说明高的信道衰落系数虽然会使网络中用户的接收有用信号的功率降低,但同时对接收干扰信号影响更为严重,
从而在干扰受限的信道条件下,大的信道衰落系数的遍历容量性能更优。
对网络的覆盖率性能进行仿真分析并与之前分析得到的理论值进行比较,
理论值和仿真值基本重合,验证了理论分析的正确性。
对密集热点区域中大尺度衰落系数、基站密度、用户的分散程度对网络性能的影响进行了方针分析,
得到覆盖率是大尺度衰落系数的增函数,是基站密度,用户分散程度的减函数。
对网络中单位面积频谱效率随密度变化的曲线进行了仿真,曲线表明网络的单位面积频谱效率是基站密度的增函数,但增长率越来越低。

(2)~~对密集热点区域无线网络进行优化,根据对密集热点区域无线网络的性能分析可知,
网络中的边缘用户受到微基站的强烈干扰,因此边缘用户的性能较差。
为了减少边缘用户,优化网络的性能,提出了网络干扰管理架构,该干扰管理架构分为两个部分,
第一个部分对基站进行分簇,提出两种微基站的分簇算法,
分别为基于深度优先搜索的微基站分簇算法和基于~k~-~均值的微基站分簇算法。
基于深度优先搜索的基站分簇算法,将距离低于门限~$\tau$~的微基站进行联合,作为一簇。
由于相距低于门限~$\tau$~的基站联合在了一起,边缘用户的数量大大降低了。
基于~k~-~均值的基站分簇算法,将基站均匀的分配在不同的簇中,簇内的边缘区域得到了消除。
第二个部分对簇内进行干扰消除,提出采用基于~ZFBF~的簇内干扰消除算法,
该算法通过将待发送的信号在空域中对齐的方法对簇内的用户实现干扰消除。
对提出的干扰管理优化算法的性能进行仿真分析,得到的结果表明,应用了干扰管理算法的网络覆盖率性能
有明显的提升,由于单位面积频谱效率是覆盖率的积分形式,该结果也表明应用干扰管理算法对网络的性能进行优化,
也大大提升了网络的单位面积频谱效率。

本文还存在一些问题尚未解决。可以根据用户的位置和信道状态信息对簇内用户进行用户选择,进一步提升网络的性能。
可以增加网络的基站睡眠与唤醒机制,降低用户的干扰和网络的总的能耗。这些问题可以作为后续的研究方向。

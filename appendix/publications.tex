% !Mode:: "TeX:UTF-8"

\defaultfont
\BiAppendixChapter{攻读\cxuewei 学位期间发表的论文及其他成果} {Papers
published in the period of PH.D. education}
\setlength{\parindent}{0em}
\textbf{(一)发表的学术论文}
\begin{publist}
\item Cao Fengfeng, Yu Qiyue, Xiang Wei, Meng Weixiao. BER Analysis of Physical-Layer Network Coding in the AWGN Channel with Burst Pulses[J]. IEEE Access, 2016, 4:9958-9968.~(SCI~收录,IDS号为~EU6IN,IF=1.270~)
\item 曹凤凤, 智小楠, 于启月. 基于MFSK的物理层网络编码在衰落信道中的性能分析[J].中国科技论文在线, 2016.
\item 何东杰, 曹凤凤, 于启月. 多址接入信道容量边界的理论分析[J].中国科技论文在线, 2016.
\end{publist}

\textbf{(二)申请及已获得的专利}
\begin{publist}
\item 于启月,曹凤凤,蔺泓如,何东杰,周永康,孟维晓. 一种基于~LDPC~码的~MIMO~传输分集方法:中国,201710457361.2[P]. 2015-06-16.
\item 于启月,宋天鸣,曹凤凤,孟维晓,何东杰. 物理层网络编码同步方法:中国,201510823133.3[P]. 2015-11-23.
\end{publist}

\textbf{(三)参与的科研项目及获奖情况}
\begin{publist}
\item 参与国家自然科学基金《基于近世代数的~SCMA~多维码本设计及其宽带无线传输理论》~(~项目编号:61671171~)
\item 参与国家自然科学基金《联合数据区分编码理论的有记忆物理层网络编码方法》~(~项目编号:61201148~)
\end{publist}
\vfill
\hangafter=1\hangindent=2em\noindent

\setlength{\parindent}{2em}

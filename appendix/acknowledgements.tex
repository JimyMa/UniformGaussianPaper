% !Mode:: "TeX:UTF-8"

\BiAppendixChapter{致\quad 谢}{Acknowledgements}

感谢陈晓华教授,作为我的导师,
您不但教我们做学问,您还教诲我们做人的道理。
钦佩您不服输的精神、正派的人格、高超的学术水平和认真的态度。

感谢孟维晓教授,得益于您的帮助才让我顺利地开启了这段学术生涯。
您品格崇高、平易近人、思维敏捷、大将之风,让我非常佩服。

感谢于启月教授,遇到您是我大学以来最幸运的一件事,这或许也会一直持续下去。
我会深遵您的教诲,做一个知行合一,外圆内方的人。

感谢韩帅老师,您告诉我结果有些时候并不重要,重要的是在这个过程中积累的能力。
这在我论文撰写的后期给了我鼓舞。

感谢何晨光老师在讨论会上对我的指导。

感谢陈淑怡师姐和赵天宇师姐,在小组讨论中给我提出非常宝贵的意见。

感谢孟老师大课题组中的所有人,我会继续努力,不会让大家失望。


感谢 Andrew Jeffery 先生,您的那篇惊艳的文章\citeup{ATractable} 深深地影响了我,
您文章中的公式 (2) 和 (3) 真是神来之笔。

感谢 David Mackay 先生,你的著作 \citeup{InfoTheoryInference} 是我硕士期间最喜欢的一本著作之一。
我在看书中第 22 章时想到了我毕业论文的一个重要的创新点。
除此之外,我还钦佩您知识的渊博,没有您或许 LDPC 码现在还束之高阁,而这只是您
生命中的一小部分,您还在物理学、机器学习、深度学习等领域颇有建树,
甚至还关心世界能源的可持续发展问题。您的世界太丰富了,我也不应该着眼在一处,
您给了我把文章写完的勇气。得知您于 2016 年 4 月 14 日英年早逝的消息我非常悲痛,
我会努力把自己变成像您一样的学识渊博、有趣、真正爱科学的人。

如果有一个叫“憎恨”的章节与致谢相对应,我最先想到的就是 Elwyn Berlekamp 先生, 您的著作
《 Algebraic Coding Theory 》让我痴迷,书中的第 7 章我看了好几遍才粗懂其中的
奥义。真正有点看懂的那天我记忆犹新,那天我正好在当助教,老师让我看不懂就再看一遍,
那一遍我终于看懂了,就像看了一本很精彩的小说一样。然而看懂了并没有任何的用处,如果时间回到两年前,
我不想去碰那本书,毕竟世界上有意思的“小说”那么多,何必看没有什么价值的这一本。

书不成字,纸短情长。最后感谢帮助过我和默默支持我的所有人。

谨以致谢,纪念这被偷走的两年。
